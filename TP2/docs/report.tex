\documentclass[11pt]{article}

\usepackage[english]{babel}
\usepackage{indentfirst}
\usepackage{graphicx}
\usepackage{subfig}
\usepackage[section]{placeins}

\usepackage{xcolor}
\usepackage{listings}
\usepackage{float}
\usepackage{capt-of}

\definecolor{mGreen}{rgb}{0,0.6,0}
\definecolor{mGray}{rgb}{0.5,0.5,0.5}
\definecolor{mPurple}{rgb}{0.58,0,0.82}
\definecolor{backgroundColour}{rgb}{0.95,0.95,0.92}

\lstdefinestyle{CStyle}{
    backgroundcolor=\color{backgroundColour},   
    commentstyle=\color{mGreen},
    keywordstyle=\color{magenta},
    numberstyle=\tiny\color{mGray},
    stringstyle=\color{mPurple},
    basicstyle=\footnotesize,
    breakatwhitespace=false,         
    breaklines=true,                 
    captionpos=b,                    
    keepspaces=true,                 
    numbers=left,                    
    numbersep=5pt,                  
    showspaces=false,                
    showstringspaces=false,
    showtabs=false,                  
    tabsize=2,
    language=C
}

\begin{document}

\begin{titlepage}
	\begin{center}
		\vspace*{1cm}
		
		\Large
		\textbf{Puzzle 2D - Wrong Products}
		
		\vspace{0.5cm}
		\large
		Segundo Projeto
		
		\vspace{1.5cm}
		
		\textbf{Hugo Miguel Monteiro Guimarães}\\
		\textbf{Beatriz Costa Silva Mendes}
		
		\vspace{5cm}
		
		Trabalho realizado no âmbito da\\
		Unidade Curricular de Programação Lógica
		
		\vspace{0.8cm}
	
		\includegraphics[width=0.4 \textwidth]{feup_logo.png}
		
		\vspace{1.5cm}		
		
		\large
		Mestrado Integrado em Engenharia Informática e Computação\\
		Faculdade de Engenharia da Universidade do Porto\\
		Porto\\
		28	 de dezembro de 2020
	
	\end{center}
\end{titlepage}


\pagebreak
\tableofcontents

\pagebreak


\section{Resumo} Contextualizar e resumir o trabalho, salientando o objetivo, o método
utilizado e referindo os principais resultados e conclusões.

Este trabalho foi desenvolvido no âmbito da Unidade Curricular de Programação em Lógica, através do sistema SICStus Prolog, e o seu objetivo foi resolver um problema de decisão/otimização utilizando Programação de lógica com restrição sobre domínios finitos. O problema escolhido foi \textbf{\emph{Wrong Products}} e tem como objetivo colocar números numa grelha de modo a que cada linha e coluna contenha apenas 2 dígitos e que o produto entre os mesmos corresponda a um determinado valor no exterior da grelha, com variação de uma unidade.


\section{Introdução} Descrição da contextualização, objetivos e motivação do
trabalho (não do ponto de vista de relatório da UC, mas de um ponto de vista
computacional / aplicacional), referência sucinta ao problema em análise (idealmente,
incluindo referências a outros trabalhos sobre o mesmo problema e sua abordagem), e
descrição sucinta da estrutura do resto do artigo

O Problema de Otimização escolhido ,\emph{Wrong Products}, tem como objetivo colocar números numa grelha de modo a que cada linha e coluna contenha apenas 2 dígitos e que o produto entre os mesmos corresponda a um determinado valor no exterior da grelha, com variação de uma unidade.

Este problema tem como objetivo a implementação de uma grelha matricial com restrições sobre linhas, colunas e toda a malha utilizando programação de lógica com restrições sobre domínios finitos.

\bigskip

Este artigo possui a seguinte estrutura:

\begin{description}

\item[Descrição do Problema:] Descrever com detalhe o problema de
otimização ou decisão em análise, incluindo todas as restrições envolvidas.

\item[Abordagem:] Descrever a modelação do problema como um PSR / POR:

\begin{description}

\item[Variáveis de Decisão] Descrever as variáveis de decisão e
respectivos domínios, assim como o seu significado no contexto do problema em análise.

\item[Restrições] Descrever as restrições rígidas e flexíveis do problema e a
sua implementação utilizando o SICStus Prolog.

\item[Visualização da Solução:] Explicar os predicados que permitem
visualizar a solução em modo de texto.

\end{description}


\item[Experiências e Resultados:]

\begin{description}

\item[Análise Dimensional] Incluir exemplos de execução em instâncias do problema com
diferentes dimensões e analisar os resultados obtidos.

\item[Estratégias de Pesquisa] Devem ser testadas diferentes estratégias de pesquisa
(heurísticas de escolha de variável e de valor), comparando os resultados obtidos.
Devem ser usadas formas convenientes para apresentar os resultados (tabelas/gráficos).

\end{description}


\item[Conclusões e Trabalho Futuro:] Que conclusões retira
deste projeto? O que mostram os resultados obtidos? Quais as vantagens e limitações da
solução proposta? Como poderia melhorar o trabalho desenvolvido?

\item[Referências] Fontes bibliográficas usadas, incluindo Livros, artigos, páginas
Web, entre outros, e apresentados segundo o formato sugerido no template.

\item[Anexo] Resultados detalhados, e outros elementos úteis que não sejam essenciais
ao relatório (não contabilizados para o limite de páginas).

\end{description}






\section{Descrição do Problema} Descrever com detalhe o problema de
otimização ou decisão em análise, incluindo todas as restrições envolvidas.


\section{Abordagem} Descrever a modelação do problema como um PSR / POR:

\subsection{Variáveis de Decisão} Descrever as variáveis de decisão e
respetivos domínios, assim como o seu significado no contexto do problema em análise.

\subsection{Restrições} Descrever as restrições rígidas e flexíveis do problema e a
sua implementação utilizando o SICStus Prolog

\section{Visualização da Solução} Explicar os predicados que permitem
visualizar a solução em modo de texto.

\section{Experiências e Resultados}

\subsection{Análise Dimensional} Incluir exemplos de execução em instâncias do problema com
diferentes dimensões e analisar os resultados obtidos.

\subsection{Estratégias de Pesquisa} Devem ser testadas diferentes estratégias de pesquisa
(heurísticas de escolha de variável e de valor), comparando os resultados obtidos.
Devem ser usadas formas convenientes para apresentar os resultados (tabelas/gráficos).

\section{Conclusões e Trabalho Futuro} Que conclusões retira
deste projeto? O que mostram os resultados obtidos? Quais as vantagens e limitações da
solução proposta? Como poderia melhorar o trabalho desenvolvido?


\section{Referências} Fontes bibliográficas usadas, incluindo Livros, artigos, páginas
Web, entre outros, e apresentados segundo o formato sugerido no template.

\section{Anexo} Resultados detalhados, e outros elementos úteis que não sejam essenciais
ao relatório (não contabilizados para o limite de páginas).




\end{document}