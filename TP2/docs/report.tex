\documentclass[11pt]{article}

\usepackage[english]{babel}
\usepackage{indentfirst}
\usepackage{graphicx}
\usepackage{subfig}
\usepackage[section]{placeins}

\usepackage{xcolor}
\usepackage{listings}
\usepackage{float}
\usepackage{capt-of}

\definecolor{mGreen}{rgb}{0,0.6,0}
\definecolor{mGray}{rgb}{0.5,0.5,0.5}
\definecolor{mPurple}{rgb}{0.58,0,0.82}
\definecolor{backgroundColour}{rgb}{0.95,0.95,0.92}

\lstdefinestyle{CStyle}{
    backgroundcolor=\color{backgroundColour},   
    commentstyle=\color{mGreen},
    keywordstyle=\color{magenta},
    numberstyle=\tiny\color{mGray},
    stringstyle=\color{mPurple},
    basicstyle=\footnotesize,
    breakatwhitespace=false,         
    breaklines=true,                 
    captionpos=b,                    
    keepspaces=true,                 
    numbers=left,                    
    numbersep=5pt,                  
    showspaces=false,                
    showstringspaces=false,
    showtabs=false,                  
    tabsize=2,
    language=C
}

\begin{document}

\begin{titlepage}
	\begin{center}
		\vspace*{1cm}
		
		\Large
		\textbf{Puzzle 2D - Wrong Products}
		
		\vspace{0.5cm}
		\large
		Segundo Projeto
		
		\vspace{1.5cm}
		
		\textbf{Hugo Miguel Monteiro Guimarães}\\
		\textbf{Beatriz Costa Silva Mendes}
		
		\vspace{5cm}
		
		Trabalho realizado no âmbito da\\
		Unidade Curricular de Programação Lógica
		
		\vspace{0.8cm}
	
		\includegraphics[width=0.4 \textwidth]{feup_logo.png}
		
		\vspace{1.5cm}		
		
		\large
		Mestrado Integrado em Engenharia Informática e Computação\\
		Faculdade de Engenharia da Universidade do Porto\\
		Porto\\
		28	 de dezembro de 2020
	
	\end{center}
\end{titlepage}


\pagebreak
\tableofcontents

\pagebreak


\section{Resumo} 
Este trabalho foi desenvolvido no âmbito da Unidade Curricular de Programação em Lógica, através do sistema SICStus Prolog, e o seu objetivo foi resolver um problema de decisão utilizando Programação de lógica com restrição sobre domínios finitos. O problema escolhido foi \textbf{\emph{Wrong Products}} e tem como objetivo colocar números numa grelha de modo a que cada linha e coluna contenha apenas 2 dígitos e que o produto entre os mesmos corresponda a um determinado valor no exterior da grelha, com variação de uma unidade.


\section{Introdução}

O Problema de Otimização escolhido ,\emph{Wrong Products}, tem como objetivo colocar números numa grelha de modo a que cada linha e coluna contenha apenas 2 dígitos e que o produto entre os mesmos corresponda a um determinado valor no exterior da grelha, com variação de uma unidade.

Este problema tem como objetivo a implementação de uma grelha matricial com restrições sobre linhas, colunas e toda a malha utilizando programação de lógica com restrições sobre domínios finitos.

\bigskip

Este artigo possui a seguinte estrutura:

\begin{description}

\item[Descrição do Problema:] Descrição com detalhe do problema de
otimização ou decisão em análise, incluindo todas as restrições envolvidas.

\item[Abordagem:] Descrição da modelação do problema como um Problema de Satisfação de Restrições(PSR)

\begin{description}

\item[Variáveis de Decisão] Descrição das variáveis de decisão e
respetivos domínios, assim como o seu significado no contexto do problema em análise.

\item[Restrições] Descrição das restrições rígidas e flexíveis do problema e a
sua implementação utilizando o SICStus Prolog.

\item[Visualização da Solução:] Explicação dos predicados que permitem
visualizar a solução em modo de texto.

\end{description}


\item[Experiências e Resultados] Análise de problema e resultados obtidos

\begin{description}

\item[Análise Dimensional] Exemplificação da execução de instâncias do problema com
diferentes dimensões e análise dos resultados obtidos.

\item[Estratégias de Pesquisa] Descrição de diferentes estratégias de pesquisa, comparando os resultados.

\end{description}


\item[Conclusões e Trabalho Futuro:] Conclusões obtidas pela realização do trabalho

\item[Referências] Referências bibliográficas utilizadas

\item[Anexo] Anexos de resultados úteis para a resolução do problema.

\end{description}



\section{Descrição do Problema}

Wrong Products é um problema de decisão. Este problema pretende descobrir se é possível colocar números numa grelha de modo a que cada que cada número apareça uma única vez, e que cada linha e coluna contenha unicamente dois números, e que o seu produto corresponda a uma unidade acima ou abaixo de um determinado valor no exterior da grelha associado à respetiva linha ou coluna.


\section{Abordagem}

Wrong Products é um Problema de Satisfação de Restrições(PSR) e é modelado por:

\subsection{Variáveis de Decisão}

Na resolução deste problema, é necessário criar as seguintes variáveis:

\begin{description}

\item[Length - ] Tamanho da grelha, indicando o número de linhas e colunas. Estes valores são iguais dado que a grelha corresponde a uma matriz quadrada. O valor de \emph{length} é passado como argumento do predicado, pelo que o seu domínio varia conforme o pretendido pelo utilizador.

\item[RowValues - ] Lista com os valores iniciais cuja diferença entre o produto dos 2 valores da respetiva linha seja 1.

\item[ColValues - ] Lista com os valores iniciais cuja diferença entre o produto dos 2 valores da respetiva coluna seja 1.

RowValues e ColValues possuem o mesmo domínio, todos os valores inteiros positivos que não sejam superiores ao produto entre os 2 maiores valores permitidos, que corresponde a:

\[(Length *2)*(Length * 2 - 1)\]

\item[ListOfLists - ] Lista de Listas contendo a grelha do problema. Pode ser interpretada como uma representação matricial do problema. Contém valores inteiros a serem multiplicados em cada linha e coluna, e zeros que correspondem a casas vazias.

\item[Transpose - ] Lista de Listas contendo a matriz transposta da grelha do problema. Tem como propósito facilitar a aplicação de restrições às colunas da grelha do problema.

\item[Table - ] Forma achatada de ListOfLists. tem como objetivo permitir a aplicação de restrições sobre toda a grelha. 

Dado que apenas podem existir 2 números por linha e coluna, o domínio de ListOfLists, Transpose e Table é o mesmo e corresponde a todos os valores inteiros entre 0 e \(2*Length\).



\end{description}

\subsection{Restrições}

\textbf{ all\_distinct\_except\_0(List)} Permite aplicar uma restrição a todos os valores da grelha, de modo a que, tal como o nome indica, todos os valores sejam distintos exceto os zeros. Desta forma é possível garantir que não há valores semelhantes e, simultaneamente, utilizar o valor zero como representação de uma casa não ocupada.

\bigskip

\textbf{line\_restriction(List,Amount)} Recebe como argumentos uma lista de listas e um valor inteiro \emph{amount}, e restringe a quantidade de zeros em cada linha de uma Lista de Listas, consequentemente garantindo que existem apenas 2 valores em cada linha. Este predicado é evocado duas vezes, a primeira para \emph{ListOfList}, e a segunda para \emph{Transpose}, de modo a que a restrição seja aplicada quer a linhas quer a colunas.

\bigskip

\textbf{multiplication\_restriction(ListOfLists,List)} Recebe uma Lista de Listas e verifica se o produto dos valores diferentes de zero de cada linha difere uma unidade em relação ao respetivo valor de uma Lista. Este predicado é evocado duas vezes, a primeira para \emph{ListOfList} e \emph{RowValues}, e a segunda para \emph{Transpose} e \emph{Colvalues}, garantindo que tanto o produto de uma linha como o de uma coluna cumprem a restrição da multiplicação enunciada anteriormente.


\section{Visualização da Solução}

Ao selecionar a opção de execução do nosso programa (se quer ver um dos problemas 
\textit{default} ou resolver um problema gerado aleatoriamente), primeiro é 
mostrado um problema por resolver, ou seja, com os espaços dentro da matriz em branco.
Este problema por resolver é obtido através do predicado \textbf{display}, que além
de demonstrar a solução, faz também um \textit{refresh} ao ecrã.
Posteriormente, quando o utilizador pede a solução do problema, esta é apresentada graças
ao predicado \textbf{displayWithoutClean}, que faz o mesmo que o predicado \textbf{display},
no entanto não faz um \textit{refresh} ao ecrã.

Ambos os predicados mencionados anteriormente imprimem os valores das colunas inicialmente
e imprimem recursivamente os valores da tabela juntamente com os valores por cada linha.

\section{Experiências e Resultados}

\subsection{Análise Dimensional} Incluir exemplos de execução em instâncias do problema com
diferentes dimensões e analisar os resultados obtidos.

Embora Wrong Products seja um problema apresentado sobre a forma de uma matriz quadrada de dimensão 4, resolvemos este problema de decisão dinamicamente, sendo possível fazer variar a dimensão da grelha e obter as soluções existentes, se possível.

Deste modo, executamos o problema com várias dimensões e verificamos que ...(acrescentar aqui algo após se terminaro projeto e fizer os gráficos)

\begin{figure}[h!]
	\begin{center}
		\includegraphics[width=0.8\textwidth]{tabela-normal.jpg}
		\caption{Tabela com Média de Tempos de Execução}
	\end{center}
\end{figure}

\begin{figure}[h!]
	\begin{center}
		\includegraphics[width=0.8\textwidth]{grafico-normal.jpg}
		\caption{Grafico com Média de Tempos de Execução}
	\end{center}
\end{figure}

\subsection{Estratégias de Pesquisa} Devem ser testadas diferentes estratégias de pesquisa
(heurísticas de escolha de variável e de valor), comparando os resultados obtidos.
Devem ser usadas formas convenientes para apresentar os resultados (tabelas/gráficos).

Temos que criar um gerador e arranjar varias heurísticas que produzam uma solução

\section{Conclusões e Trabalho Futuro} 


\section{Referências} 



\section{Anexo} 


\end{document}